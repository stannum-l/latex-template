\documentclass[10pt,oneside,numbers=endperiod,a4paper]{scrartcl}
\usepackage{stannum}
\pagestyle{plain.scrheadings}
% \everymath{\displaystyle}

\begin{document}
\begin{Problem}
    Let $a, b \in \RR$, not both $0$. Find $c, d \in \RR$ such that $1/(a + bi) = c + di$.
\end{Problem}

\begin{Solution*} $c := \frac a{a^2+b^2};\ d:=\frac{-b}{a^2+b^2}.$
    \[ \frac 1{a+bi} \cdot \frac{a-bi}{a-bi} = \frac{a-bi}{a^2 + b^2} = 
    \underbrace{\frac{a}{a^2+b^2}}_{=:c} + 
    \underbrace{\frac{-b}{a^2+b^2}}_{=:d} i.\]
\end{Solution*}

\vspace{0.5cm}

\begin{Problem}
    Prove that $-(-v)=v$ for every $v \in V$.
\end{Problem}

\begin{Proof*}
    Let $v \in V$ be given. Consider the expression $v + (-v) + (-(-v))$.
    \begin{align*}
        v + \underbrace{(-v) + (-(-v))}_{=0} &= v + 0 = v \\
        \underbrace{v + (-v)}_{=0} + (-(-v)) &= 0 + (-(-v)) = -(-v).
    \end{align*}
    Thus, $v = -(-v)$.
\end{Proof*}

% Problem 1B.2
\begin{Problem}
    Prove that if $a \in \F$, $v \in V$, and $av = 0$, then $a = 0$ or $v = 0$.
\end{Problem}

\begin{Proof}
    Suppose $a \in \F$, $v \in V$ with $av = 0$. Assume $a \ne 0 \ne v$.
    Since $a \neq 0$, it has a multiplicative inverse $a\inv$ so $aa\inv = a\inv a = 1$.
    Then:
    \begin{align*}
        a\inv av &= a\inv (av) = a\inv 0 = 0, \\
        a\inv av &= (a\inv a)v = 1v = v \neq 0.
    \end{align*}
    This is a contradiction.
\end{Proof}

\begin{Problem} Determine if the following is a subspace of $\F^3$.
    \begin{enum}
        \item $\set{(a,b,c) \in \F^3 : a + 2b + 3c = 4}$.
        \item $\set{(a,b,c) \in \F^3 : abc = 0}$.
        \item $\set{(a,b,c) \in \F^3 : a = 5c}$.
    \end{enum}
\end{Problem}

\begin{Solution*}
    \begin{enum}    
        \item No. Consider element $(4,0,0)$ and the scalar multiplier $2$.
        \item No. Consider $(0,1,2) + (1,0,2) = (1,1,4)$.
        \item Yes. 
    \end{enum}
\end{Solution*}
    

\begin{Problem}
    Give an example of $\emptyset \ne U \subseteq \RR^2$ such that $U$ is closed under addition and under
    taking additive inverse, but $U$ is not a subspace of $\RR^2$.
\end{Problem}

\begin{Solution*}
    Consider $U := \set{(x,y) \in \ZZ^2}$. Note $\ZZ$ is closed under addition and under taking additive inverse. However, take $c \in \RR \varsetminus \ZZ$, e.g. $c = \sqrt 2$,
    then $\sqrt 2(1,1) = (\sqrt 2, \sqrt 2) \notin U$.
\end{Solution*}

\begin{Problem}
    Give an example of $\emptyset \ne U \subseteq \RR^2$ such that $U$ is closed under scalar multiplication
    but $U$ is not a subspace of $\RR^2$.
\end{Problem}

\begin{Solution*}
    Consider $U := \set{(x,0) : x \in \RR} \cup \set{(0,y) : y \in \RR}$. Given $c \in \RR$.
    $c(x,0) = (cx,0) \in U$, and $c(0,y) = (0,cy) \in U$, so it is closed under scalar multiplication.
    However, $(1,0) + (0,1) = (1,1) \notin U$, so it's not a subspace.
\end{Solution*}

\begin{Problem}
    Let $U, V$ be subspace of $W$. Prove $U \cap V$ is a subspace of $W$ iff $U \subseteq V$ or $V \subseteq U$.
\end{Problem}

\begin{Proof*} Let vector space $W$ and subspaces $U, V$ of $W$ be given.
    
    \iffind If $U \subseteq V$, then $U \cap V = V$. If $V \subseteq U$, then $U \cap V = U$. In both cases, $U \cap V$ is a subspace by assumption.

    \ifind Suppose for a contradiction that $U \cap V$ is a subspace but neither set contains the other.
    That means, we can find $u \in U \varsetminus V$ and $v \in V \varsetminus U$.
\end{Proof*}

\begin{Problem}
    Does the operation of addition on the subspaces of $V$ have an additive identity? Which subspaces
    have additive inverses?
\end{Problem}

\begin{Solution*}
    No. The trivial subspace $\set{0}$ as $0$ is its own inverse.
\end{Solution*}

\begin{Problem*}
    Prove or give counterexample: If $U_1$, $U_2$, $W$ are subspaces of $V$ such that
    $U_1 + W = U_2 + W$, then $U_1 = U_2$.
\end{Problem*}

\begin{Problem}
    Suppose $U$ is the subspace of $P(\F)$ consisting of all polynomials $p$ of the form $p(z) = az^2 + bz^5$,
    where $a, b \in \F$. Find a subspace $W$ of $P(\F)$ such that $P(\F) = U \oplus W$.
\end{Problem}
\end{document}