\documentclass[11pt]{scrreprt}
\usepackage[sexy]{evan}
\renewcommand{\figurename}{Image}
\numberwithin{figure}{chapter}

\begin{document}
\title{The 55th International Mathematical Olympiad}
\subtitle{Cape Town, South Africa}
\author{Evan Chen}
\date{July 2014}
\maketitle

\tableofcontents

\chapter{Problems}
It appears that the protocol for 1, 2, 4, 5 being distinct subjects is still in effect.
Unfortunately, this has forced the inclusion of Problem 5 as a fake N which is really C, making this an
IMO with three combinatorics problems.
\section{Day 1}
\begin{problem}
  Let $a_0 < a_1 < a_2 \ldots$ be an infinite sequence of positive integers. Prove that there exists a unique integer $n\geq 1$ such that \[a_n < \frac{a_0+a_1+a_2+\cdots+a_n}{n} \leq a_{n+1}.\]
\end{problem}

\section{Problem 4}
Since $PB = c^2/a$ we have $P = (0 : a^2-c^2 : c^2)$, so $M = (-a^2 : {--} : 2c^2)$. Similarly $N = (-a^2 : 2b^2 : {--})$. Thus
\[ \overline{BM} \cap \overline{CN} = (-a^2 : 2b^2 : 2c^2) \]
which clearly lies on the circumcircle.


\section{Problem 5}
We'll prove the result for at most $k - \tfrac 12$ with $k$ groups.
First, perform the following optimizations.
\begin{itemize}
  \ii If any coin of size $\frac{1}{2m}$ appears twice, then replace it with a single coin of size $\frac{1}{m}$.
  \ii If any coin of size $\frac{1}{2m+1}$ appears $2m+1$ times, group it into a single group and induct downwards.
\end{itemize}
Apply this operation repeatedly until it cannot be done anymore.

Now construct boxes $B_0$, $B_1$, \dots, $B_{k-1}$.  In box $B_0$ put any coins of size $\tfrac 12$ (clearly there is at most one).
In the other boxes $B_m$, put coins of size $\frac{1}{2m+1}$ and $\frac{1}{2m+2}$ (at most $2m$ of the former and at most one of the latter).
Note that the total weight in the box is less than $1$.
Finally, place the remaining ``light'' coins of size at most $\frac{1}{2k+1}$ in a pile.

Then just toss coins from the pile into the boxes arbitrarily, other than the proviso that no box should have its weight exceed $1$.
We claim this uses up all coins in the pile. Assume not, and that some coin remains in the pile when all the boxes are saturated.
Then all the boxes must have at least $1 -\frac{1}{2k+1}$, meaning the total amount in the boxes is strictly greater than
\[ k \left( 1 - \frac{1}{2k+1} \right) > k - \tfrac 12 \]
which is a contradiction. This gets a stronger bound $k - \frac{k}{2k+1}$ than the requested $k-\tfrac 12$.

And dear IMO jury: THIS IS NOT A NUMBER THEORY PROBLEM.


\section{Problem 6}
Let $c = (6e)^{-\frac12}$. We'll show the bound $c \sqrt n$.

The main core of the proof is the following lemma.
\begin{lemma*}[Lovasz Local Lemma]
  Consider several events, each occurring with probability at most $p$, such that
  each event is depends on at most $d$ of the others. If \[ epd \le 1 \]
  then there is a nonzero probability that no events happen.
\end{lemma*}
Split the $n$ lines into $c \sqrt n$ groups of size $\frac{\sqrt n}{c}$ each, arbitrarily.
We are going to select one line from each of the groups at random to be blue.
For each of the regions $R_1$, $R_2$, \dots, $R_m$ we will consider an event $A_k$ meaning ``three of the lines bounding $R_k$ are blue'';
we designate these lines beforehand.
We will show there is a nonzero probability that no events occur.

The probability of $A_k$ is clearly at most $\left( \frac{c}{\sqrt n} \right)^3$.

For each $R_k$, we have three groups to consider. Each group consists of $\frac{\sqrt n}{c}$ lines. Each line is part of at most $2n-2$ regions.
Hence $A_k$ depends on at most $3 \cdot \frac{\sqrt n}{c} \cdot (2n-2)$ events.

Thus,
\[ e \left( \frac{c}{\sqrt n} \right)^3 \left( 3 \cdot \frac{\sqrt{n}}{c} \cdot (2n-2) \right)
  < 6ec^2 = 1 \]
and we are done by LLL.


\section{Day 1 Contest Analysis}
I start with Problem 3 for the first ten minutes and lament the nonexistent diagram. I don't get very far other than drawing an okay picture and figuring out that the ``tangent to $BD$'' is just fancy for ``the circumcenter is on $AH$''.
I consider complex numbers but it seems to messy.
Thus I decide it's a good time to use the restroom and get some water.

When I return I decide to begin working on \#2, the combinatorics.
I bash out the cases for $n=1,2,3,4,5,6,7$ before deciding that this is getting annoying.
During this time I realize (by doing $n=5$) that when $n=m^2+1$ it's trivial to find an $m \times m$ bound.

At the time I don't think much of it and decide to go kill \#1, because it's been half an hour and I feel like I should probably get something done.
Basically I write the first thing that comes to mind nicely, and it works, because of course it works, because it's a \#1. I almost flip inequalities
a bunch of times but after half an hour of scribbling and crossing things out the problem is dead and I'm like okay.
This was in retrospect sub-optimal and I should have looked at least one specific case.
Not a big deal though, \#1 is \#1.

I then return to \#2 and look at $n=5$ again, musing that the bound $k=2$ was quite weak.
Then I realize $n=4$ has answer $k=1$.
Immediately I become convinced that the answer was just $\left\lfloor \sqrt{n-1} \right\rfloor$ and that the problem was merely
finding the necessary construction when $n=m^2$, because this was a \#2 and rooks are weird so there are not that many ways
the problem can end.
It takes about half an hour of fooling around to find said configuration -- it is a helpful hint that the $m^2 \times m^2$ can be
divided into a bunch of $m \times m$ sub-squares which each necessarily have a rook in them.
Once I find the configuration I breathe a sigh of relief knowing that I have two problems, the ``didn't fail'' threshold.
I now have three hours left to address the third question.
I use this opportunity to use the restroom again.

Since this is a \#3 geometry I would like to be looking at a diagram for things to prove are true, but the condition is really rather annoying.
It takes me almost an hour to realize that in fact the circumcenter of $\triangle TCH$ lies on $AD$ -- I finally realized this when blindly
angle chasing and looking for things that are cyclic when I noticed that the tangent to $(TCH)$ at $T$ was a line perpendicular to $AD$.
Suddenly I can actually draw the diagram.

Except at this point I have the urge to bash. Specifically, I just have to show that the perpendicular bisector of $HT$ meets $AH$ in the same place, meaning I have to show $PH$ depends is symmetric in $b$, $d$, where $P$ was the intersection.
And FOR THE FIRST TIME IN FOREVER, the Law of Cosines looks like a good option.
I'm hesitant to use it since I have never ever used the Law of Cosines in an olympiad before, and the IMO is a sub-optimal time to try a technique for the first time.

But it just looked too nice: let $x=AO$, $h = AH$, $d=AB$, $b=AD$, $R = \half AC$, and $r=HO=CO=TO$, where $O$ is the circumcenter of $\triangle THC$.  Then
\[ r^2 = h^2 + x^2 - 2xh \frac{d}{2R}
  = (2R)^2 + x^2 - 2(2R)h \frac{b}{2R}. \]
Here $h = \frac{1}{2R} bd$. Subtracting the two equations cancels off the $r^2$ and $x^2$, giving us a linear equation to quickly find $x$.

I wasn't sure how to pin down $PH$, though. My first impression was to use $x$ to compute $r$, then find $AT = x-r$, and then do some computations. I had planned to use a homothety with ratio $2$ at $H$ to compute $P'$, the point on $AH$ such that $\angle P'TH = 90\dg$.
These computations quickly get hairy and I don't actually want to do them.
Most importantly, I really did not want to evaluate $\sqrt r$.

I spend another 30-60 minutes slogging around rather lost, constantly wavering between trying to find a better bash and going for the all-out slog with the time remaining,  until I finally realize that
\[ \frac{AP}{PH} = \frac{AH}{OH} = \frac{x}{r} \]
by the Angle Bisector Theorem. And so I just had to show this was symmetric, but
\[ \frac{r^2}{x^2} = 1 + \frac{(2R)^2}{x^2} - \frac{2bh}{x}. \]
You can tell by looking now that this should be really direct to compute. It shouldn't take more than 15 minutes by any contestant in the room, and shouldn't take more than 10 for a seasoned basher like me.
But the pressure of the IMO sets in around now, and I repeatedly miscomputed for the next while.

Sitting at the edge of getting an IMO3, I become so nervous that I begin visibly shaking.
I have around 70-90 minutes left, which I tell myself is more than enough.
I go to use the bathroom and come back no longer shaking -- standing up does wonders for this sort of thing.
I then re-start the bash from scratch and finish in 10 minutes.

Well\dots done with the IMO an hour early. What do you know?

I consider walking out early since I probably will never have a chance to say ``I walked out of the IMO early'' ever again, but decide against it since it's a rather mean to the other contestants.
I neatly collate my solutions and the scratch work into the problem folders. It probably was not necessary to include my failed diagrams and random wrong calculations in \#3, or any of the other problems for that matter, but I did it just in case anyways.
This inflated my four-page solution to \#3 to seventeen pages of script.

I eat a little bit of chocolate and then decide it's an excellent time to start drawing random pictures to add to the end of my solutions.
I draw a scene of a large chessboard with rooks and peace signs hanging out in distinct rows and columns, and then another picture of a dinosaur labelled $a_{2014}$ chasing after some other terms of the sequence $(a_n)$, and so on.
One of the invigilators notices me doing this and laughs.

Well, whatever. Just have to stay calm for Day 2.

\section{Day 2 Contest Analysis}
First I incinerate \#4 with barycentric coordinates.  Cool.

Then I get stuck on \#5 for three hours. It is really very slippery; there are too many things to try that do not work.
The greedy algorithm in fact fails if applied directly and some optimization is needed.
I begin the problem with an optimization of switching $\frac{1}{km} \times k$ with $\frac{1}{m}$, limiting the number of times each coin appears,
and in particular making even-numbered coins appear at most once.
I also remove any $1$'s that show up as this occurs, since one can just induct down ($99$ is just a random number, clearly\dots).
But it takes forever and a half until I find the construction; I finally noticed it while staring at the row of numbers
\[ \half \qquad \frac 13 \times 2 \quad \frac 14 \qquad \frac 15 \times 4 \quad \frac 1 6 \qquad \dots \]
and realizing I could get $k$ bins and a bunch of small $\frac{1}{2k}$ coins.
I had been playing with the $\frac{1}{2k}$ coins idea but had approached it from the wrong angle of a stronger induction hypothesis.

So anyways I end up with just about an hour for the final problem.
Miraculously fooling around with LLL gets me a $c>0$. Well. Okay.

GCC'ed again!

\chapter{Diary of Events}
\section{July 5 -- The Airplane Ride}
I'm designated the official camera-wielder for the trip based on my prolific picture-taking during the camps.

\end{document}
